%%%%%%%%%%%%%%%%%%%%%%%%%%%%%%%
\section{Fast-Ref}

%******************************
\subsection{Commit}
\begin{tabularx}{\textwidth}{lX}
    \TT{git add .}            & stages all the newly changed/editted files        \\
    \TT{git commit -m 'msg'}  & commits the staged files using the message: 'msg' \\
    \TT{git commit -am 'msg'} & equivalent the above two-commands                 \\
\end{tabularx}

%******************************
\subsection{Amend}
\begin{tabularx}{\textwidth}{lX}
    \TT{git commit -\:-amend -m 'newmsg'} & Replaces last commit to include the updates \& changes the commit msg                \\
    \TT{git add . }                       &                                                                                      \\
    \TT{git commit -\:-amend -\:-no-edit} & commits the staged files BUT using '-\:-no-edit' retains the previous commit message \\
    \TT{git log -\:-oneline}              & shows commit history in a shorter form as:                                           \\
                                          & e.g.: \texttt{f425059 (HEAD -> master, origin/master) msg}                           \\
                                          & \texttt{8f184d5 first commit}                                                        \\
    \TT{git revert 8f184d5}               & reverts to a previous commit with commit ID: 8f184d5,                                \\
                                          & This creates a new commit with the changes from the previous commit.                 \\
\end{tabularx}

%******************************
\subsection{Codespace}
To edit the files directly in GitHub (online), go to the intended repo. in GitHub, press ".". This opens the file in vscode.
You can make changes to the code and push them to the remote repo.

%******************************
\subsection{Stash}
To save your work without adding them to the staging area and creating a new commit. This allows you to save your progress and restore it whenever you need to. \par
\begin{tabularx}{\textwidth}{lX}
    \TT{git stash save new-idea} & saves your current progress by providing a name and stashing it.           \\
                                 & e.g.: \texttt{Saved working directory and index state On master: new-idea} \\
    \TT{git stash list}          & view your stash list and note the corresponding index to retrieve it.      \\
                                 & e.g.: \texttt{stash\@\{0\}: On master: new-idea}                           \\
                                 & stash of "new ideas" is saved at index 0.                                  \\
    \TT{git stash apply 0}       & retrives the stash of "new-idea" that was already saved at index 0         \\
    \TT{git branch -M main}      & renames your default branch name.                                          \\
                                 & In this case, it renames "master" to "main".                               \\
\end{tabularx}

%******************************
\subsection{Log}
\begin{tabularx}{\textwidth}{lX}
    \TT{git reflog}                                 & Views the history of checkouts                                     \\
    \TT{git log -\:-graph -\:-decorate -\:-oneline} & \texttt{git log} shows a detailes history of all the commits.      \\
                                                    & displays the changes made in multiple branches and how they merge. \\
                                                    & To make it more readable.                                          \\
\end{tabularx}


\secline
\subsection{Unstage}
\begin{tabularx}{\textwidth}{llX}
    01 & \TT{git restore -\,-staged  <file1, file2>} & Un-stages files 1,2 \\
\end{tabularx}

\subsection{Merge branches}
\begin{tabularx}{\textwidth}{llX}
    01 & \TT{git merge <master> <branch>}   & merge <branch> to master        \\
    02 & \TT{git merge <branch1> <branch2>} & merges:  <branch2> to <branch1> \\
\end{tabularx}

\subsection{Merge to master}
\begin{tabularx}{\textwidth}{llX}
    01 & \TT{git branch}          & Lists branches           \\
    02 & \TT{git checkout master} & switch to master         \\
    03 & \TT{git merge <branch>}  & merge <branch> to master \\
    04 & \TT{git log}             &                          \\
\end{tabularx}

\subsection{Del a branch}
\begin{tabularx}{\textwidth}{llX}
    01 & \TT{git branch}             & Lists branches   \\
    02 & \TT{git checkout master}    & switch to master \\
    03 & \TT{git branch -d <branch>} & Deletes branch   \\
    04 & \TT{git push origin}        &
\end{tabularx}

\subsection{Rename a branch}
\begin{tabularx}{\textwidth}{llX}
    01 & \TT{git branch -m <name>} & rename the current branch to <name>
\end{tabularx}

\begin{flushleft}\begin{tabularx}{\textwidth}{l|X}
        \TT{p}                                    & (going back to HEAD)                   \\
        \TT{git reset -\:-hard HEAD$\wedge$}      & (going back to the commit before HEAD) \\
        \TT{git reset -\:-hard HEAD}$\thicksim$ 1 & (equivalent to "$\wedge$")             \\
        %\TT{git reset -\:-hard HEAD$\tilde$2} &    (going back two commits before HEAD)
    \end{tabularx}\end{flushleft}
%
%
%%%%%%%%%%%%%%%%%%%%%%%%%%%%%%%
\section{First-Time Git Setup}
\noindent First-time set-up \& configuration for a new Git installation.
%
\begin{flushleft}\begin{tabularx}{\textwidth}{l|X}
        \TT{git config -\:-global user.name <FirstName LastName>}
         & Drop "\texttt{-\:-global}" option from these commands \\
        \TT{git config -\:-global user.email <email@example.com>}
         & to recognize you only locally.
    \end{tabularx}\end{flushleft}

\begin{flushleft}\begin{tabularx}{\textwidth}{l|X}
        \TT{git config -\:-global core.editor <emacs>}
         & To use a different text editor (from system default) for git. \\
        %
        \TT{git config -\:-list -\:-show-origin}
         & Shows settings and where they are coming from                 \\
        %
        \TT{git config -\:-list}
         & Checking the Settings that are in effect.
    \end{tabularx}\end{flushleft}
%
%
%%%%%%%%%%%%%%%%%%%%%%%%%%%%%%%
\subsection{remote repo}
\begin{flushleft}\begin{tabularx}{\textwidth}{l|X}
        \TT{git config user.name}  & Shows who it is configured to     \\
        \TT{git config user.email} & Shows the email associated to git
    \end{tabularx}\end{flushleft}


%%%%%%%%%%%%%%%%%%%%%%%%%%%%%%%
\section{GUI}
\begin{flushleft}\begin{tabularx}{\textwidth}{l|X}
        \TT{gitk} & Opens a visual commit browser (some GUI)
    \end{tabularx}\end{flushleft}

%%%%%%%%%%%%%%%%%%%%%%%%%%%%%%%
\section{help}
\begin{flushleft}\begin{tabularx}{\textwidth}{l|X}
        \TT{git -\,-version}         & Checks installed Git's version \\
        \TT{git help~|~git -\,-help} & Shows git help                 \\
        \TT{git help <command>}      &                                \\  Gives help about <command>
        \TT{git <command> -\,-help}  &
    \end{tabularx}\end{flushleft}
%
%
%%%%%%%%%%%%%%%%%%%%%%%%%%%%%%%
\section{Status}
\begin{flushleft}\begin{tabularx}{\textwidth}{l|X}
        \TT{git status}        & Checks the current state of repo \\
        \TT{git status <file>} & Checks state of specific file
    \end{tabularx}\end{flushleft}
%
%
%%%%%%%%%%%%%%%%%%%%%%%%%%%%%%%%%%%%%%%
\section{log}
\begin{tabularx}{\textwidth}{l|X}
        \TT{git reflog}                                 & Views the history of checkouts                                     \\
        \TT{git log}                                    & Shows a detailed log of commit history                             \\
        \TT{git log -\,-oneline}                        & Shows only <commentsID>~commit-comments                            \\
        \TT{git shortlog}                               & For a shorter log of commit history                                \\
        \TT{git shortlog -s}                            & Creates even much shorter log                                      \\
        \TT{git shortlog -1}                            & Shows only the last 'one' commit                                   \\
        \TT{git log -\:-graph -\:-decorate -\:-oneline} & \texttt{git log} shows a detailes history of all the commits.      \\
                                                        & displays the changes made in multiple branches and how they merge. \\
                                                        & To make it more readable.
\end{tabularx}


\begin{enumerate}\packed
    \item {\color{blue}\emph{-1}}:
    \item {\color{blue}\emph{-p}}: shows the line diff for each commit
    \item {\color{blue}\emph{-p -\:-word-diff}}: shows the word diff for each commit
    \item {\color{blue}\emph{-\:-stat}}: shows stats instead of diff details
    \item {\color{blue}\emph{-\:-name-status}}: shows a simpler version of stat
\end{enumerate}

%%%%%%%%%%%%%%%%%%%%%%%%%%%%%%%
\section{Create repo.}
Two ways to create a repository:
\subsection{\texttt{git init}}
\begin{flushleft}\begin{tabularx}{\textwidth}{l|X}
        \TT{git init <dir>} & Create new (empty) / reinitialize existing repo in <dir>  \\
        \TT{git init}       & Executing this in project-directory <dir> makes it a repo \\
    \end{tabularx}\end{flushleft}

%%%%%%%%%%%%%%%%%%%%%%%%%%%%%%%
\subsection{\texttt{clone}}
%
\TT{>> git clone <repo-url>}~~~~It is the URL to a remote git repository <repo>\\
1. Creates a local \texttt{<repo>} folder \& Initializes it as a git repository \\
3. Copies (pull-downs) all data from \texttt{repo-url} to the local folder\\
4. Automatically configures <repo> to point to the \texttt{<repo-url>}\\
5. \texttt{Checkout} to the local working directory\\
Once making change-\&-committing files, one can \texttt{git push} the changes to the remote repository at <repo-url>
\smallskip
\nl\TT{>> git clone <repo-url> <folder-name>}~~~~Same as above but local repo can be <folder-Name> (different from the remote one)
\smallskip
\nl\TT{>> git clone <repo1> <repo2>}~~~Copies a local repo-folder to a new local folder

\nl \textbf{Delete:} To remove git control delete ``.git'' from working director.


%%%%%%%%%%%%%%%%%%%%%%%%%%%%%%%%%%%%%%%
\section{Staging}
Staged-files are ready to be committed

%%%%%%%%%%%%%%%%%%%%%%%%%%%%%%%%%%%%%%%
\subsection{\texttt{add}}
\begin{flushleft}\begin{tabularx}{\textwidth}{l|X}
        \TT{git add  <file1 file2 file3>}        & Adds files 1-\:-3 to staging area          \\
        \TT{git add *.txt}                       & stages all text files                      \\
        \TT{git add .}                           & stages all text files                      \\
        \TT{git add -A} | \TT{git add -\,-a[ll]} & Adds evergitything in and beneath          \\
        \TT{git add -\,-u}                       & Adds modified files (but not the new ones)
    \end{tabularx}\end{flushleft}


%%%%%%%%%%%%%%%%%%%%%%%%%%%%%%%%%%%%%%%%
\subsection{\texttt{.gitignore}}
A hidden file (\texttt{.gitignore.txt}) in the root of repository specifies the files we do not need to keep track of their changes (e.g. *.exe files).  Note that the files already tracked by Git are not affected. To remove all files from the repository and adding them back according to the rules in .gitignore:
\nl Use: \TT{git rm -rf -\,-cached .}  \TO~~\TT{git add .}     \hfill (see Appendix~\ref{app:ignore-cmd})


%%%%%%%%%%%%%%%%%%%%%%%%%%%%%%%%%%%%%%%
\section{Un-staging}

\subsection{\texttt{restore}~/~\texttt{rm}}
\begin{flushleft}\begin{tabularx}{\textwidth}{l|X}
        \TT{git restore -\,-staged .}               & Un-stages all the staged files     \\
        \TT{git restore -S .}                       & Shorten form                       \\
        \TT{git rm -rf -\,-cached .}                & Un-stages all staged files~~\hl{+} \\
        \TT{git restore -\,-staged  <file1, file2>} & Un-stages files 1,2                \\
        \TT{git rm -\,-cached <file>}               & Un-stages  \texttt{<file>}         \\
    \end{tabularx}\end{flushleft}

%%%%%%%%%%%%%%%%%%%%%%%%%%%%%%%%%%%%%%%
\section{Delete}
\subsection{\texttt{rm}}
\begin{flushleft}\begin{tabularx}{\textwidth}{l|X}
        \TT{git rm <files>}
         & Deletes from both the working directory and staged area.                               \\
         & {\textcolor{red} {\textbf{Note:~}} you may lose all the changes (even the good ones):}
    \end{tabularx}\end{flushleft}

%%%%%%%%%%%%%%%%%%%%%%%%%%%%%%%%%%%%%%%
\section{Commit}
%%%%%%%%%%%%%%%%%%%%%%%%%%%%%%%%%%%%%%%
%
\begin{flushleft}\begin{tabularx}{\textwidth}{l|X}
        \TT{git commit}
         & \small{Commits staged files \& asks for "comment message" to the commit}    \\
        %
        \TT{git commit -m {\footnotesize 'msg'}}
         & \small{Commits \& (\TT{switch '-m'}) simultaneously adds a commit message}  \\
        %
        \TT{git commit -\:-all [|-a]}
         & \small{Commits all the file in the staged area and asks for the comment}    \\
        %
        \TT{git commit -am 'msg'}
         & \small{Adds modified files to stage, commits them, and adds commit message}
    \end{tabularx}\end{flushleft}


%%%%%%%%%%%%%%%%%%%%%%%%%%%%%%%%%%%%%%%
\subsection*{Un-commit}
\begin{flushleft}\begin{tabularx}{\textwidth}{l|X}
        \TT{git commit -\:-amend -m "new-msg"}
         & \small{Replaces the last commit of the current branch with the current staged files and replace its commit-msg with the "new-msg"; as last commit never happened.}                                              \\
        \TT{git commit -\:-amend}
         & \small{If no changes since last commit (e.g. immediately after a commit), This only changes the commit-msg. It opens the last commit-message in the editor for editing, any change overwrites last commit-msg.}
    \end{tabularx}\end{flushleft}

%%%%%%%%%%%%%%%%%%%%%%%%%%%%%%%%%%%%%%%
\subsection*{Reset Author}
\begin{flushleft}\begin{tabularx}{\textwidth}{l|X}
        \TT{git commit -\,-amend -\,-reset -\,-author}
         & Amends commit author \& author date to the committer                                \\
        \TT{git commit -\,-amend -\,-author=\footnotesize{"Author Name <email>"}}
         & Amends commit author with given author name \& email, author date remains unchanged \\
        \TT{git commit -\,-amend -\,-date={\footnotesize "2021-04-13T14:59:10"}}
         & Amends the commit date (use ISO 8601 format for convenience)
    \end{tabularx}\end{flushleft}




%%%%%%%%%%%%%%%%%%%%%%%%%%%%%%%%%%%%%%%
\section{\texttt{branch}}
To create, rename, delete, etc. of a branch
\begin{flushleft}\begin{tabularx}{\textwidth}{l|X}
        \TT{git branch}                  & Shows (list) both local branches                                      \\
        \TT{git branch <name>}           & Create a new (local) branch from the current Head (i.e. last commit)  \\
        \TT{git branch -m <new-name>}    & Rename the current branch to "new-name"                               \\
        \TT{git branch -d <name>}        & Delete this branch, This do not delete if branch has unmerged changes \\
        \TT{git branch -D <branch-name>} & Force delete this branch, even if it has unmerged changes             \\
        \TT{git branch -\,-a}            & Shows (list) both remote \& local tracking branches                   \\
        \TT{git branch -\,-r}            & Shows remote tracking branches
    \end{tabularx}\end{flushleft}


%%%%%%%%%%%%%%%%%%%%%%%%%%%%%%%%%%%%%%%
\section{Branch - \texttt{checkout}}
Takes to branch
\begin{flushleft}\begin{tabularx}{\textwidth}{l|X}
        \TT{git checkout -b <name>}       & Creates a new-branch <name> from current Head \& then, checkout to it ~~~\hl{+} \\
        \TT{git checkout master}          & Switches to branch "master" (github)                                            \\
        \TT{git checkout <branch-name>}   & Switches to <branch-name>~~~\hl{+}                                              \\
        \TT{git checkout <remote-branch>} &                                                                                 \\
        \TT{git checkout -\,- <files>}    & Discards all the changes in <files> and restore it from the staged-version      \\
        \TT{git checkout -\,-detach}      & Detaches Head from current branch                                               \\
    \end{tabularx}\end{flushleft}
\noindent "\textbf{checkout}" is the act of switching between different versions of a target entity.


%%%%%%%%%%%%%%%%%%%%%%%%%%%%%%%%%%%%%%%
\section{\texttt{diff}: compare}
Compares and shows differences between two instances.
\begin{flushleft}\begin{tabularx}{\textwidth}{l|X}
        \TT{git diff}                                       & Compares and shows differences between Working-Directory vs. Staging-Area                                                  \\
        \TT{git diff <file>}                                & Compares and shows modifications in current \texttt{file} in working-directory compared with \texttt{file} in staging-area \\
        \TT{git diff -\,-staged}                            & Compares all the files in staging area with the last committed ones                                                        \\
        \TT{git diff HEAD}                                  & Working-Area vs Last-Commit                                                                                                \\
        \TT{git diff <branch-name>}                         & Shows diff between branch and working-directory                                                                            \\
        \TT{git diff {\footnotesize <branch-1> <branch-2>}} & Shows diff between two branches
    \end{tabularx}\end{flushleft}

%%%%%%%%%%%%%%%%%%%%%%%%%%%%%%%%%%%%%%%
\section{Cleaning Working-Directory}
\begin{flushleft}\begin{tabularx}{\textwidth}{l|X}
        \TT{git clean -\,-n} & See what would be done by \texttt{git clean} command (Dry-Run of clean Operation)  \\
        \TT{git clean -\,-1} & Interactively(?) removes un-tracked files from repository (Remove un-traced files)
    \end{tabularx}\end{flushleft}

\begin{verbatim}
git restore -\:-staged .
Or simply you can

git restore -S .
\end{verbatim}



%%%%%%%%%%%%%%%%%%%%%%%%%%%%%%%%%%%%%%%
\section{\texttt{git reset}}
%%%%%%%%%%%%%%%%%%%%%%%%%%%%%%%%%%%%%%%
\subsection{Roll-back to a previous commit}
\begin{flushleft}\begin{tabularx}{\textwidth}{l|X}
        \TT{git reset -\,-hard HEAD} & Reverts working copy to the HEAD (most recent commit) \hl{+}
    \end{tabularx}\end{flushleft}


\begin{flushleft}\begin{tabularx}{\textwidth}{l|X}
        \TT{git reflog}           & Shows current commit history or use \TT{git log -\,-oneline}           \\
        \TT{git reset <commitId>} & Resets master to the commit <commitID> (absolute address).             \\
                                  & e.g., commit \hl{0766c05}3c0ea2035e90f504928f8df3c9363b8bd             \\
        \TT{git reset current\~2} & Resets master to 2 commit before the current commit (relative address) \\
    \end{tabularx}\end{flushleft}



\begin{flushleft}\begin{tabularx}{\textwidth}{l|X}
        \TT{git reset}              &
        \begin{enumerate}\packed
            \item Removes everything from staging area
            \item Resets every modified files in working-space to the latest commit
            \item Brings them back to the working-area
            \item[] {\textcolor{red} {\textbf{Note:~}} you may lose all the changes (even the good ones):}
        \end{enumerate}                                            \\
        \TT{git reset <files>}      & Un-stages "\texttt{<files>}"  from the indexing~\to~Reset to the latest commit~\to~Leaves them in the working-area \\
        \TT{git reset path/to/file} & Un-stages files in "\texttt{path/to/file}" folder from the indexing, $\dots$ as above
    \end{tabularx}\end{flushleft}

%%%%%%%%%%%%%%%%%%%%%%%%%%%%%%%%%%%%%%%
\subsection{\texttt{git reset HEAD}}
\begin{flushleft}\begin{tabularx}{\textwidth}{l|X}
        \TT{git reset HEAD -\:- <files>}      & Un-stages "\texttt{<files>}"  from the index                                         \\
        \TT{git reset HEAD -\,- path/to/file} & Un-stages files in "\texttt{path/to/file}" folder from the index                     \\
        \TT{git reset HEAD -\,- .}
                                              & Un-stages from the indexing all the files recursively and so forth to the subfolders
    \end{tabularx}\end{flushleft}


\section{Remote Repository}
%%%%%%%%%%%%%%%%%%%%%%%%%%%%%%%%%%%%%%%%
\subsection{Git-Hub}
Given you have a GitHub account:
\begin{enumerate}\packed
    \item log-in to:~~\url{https://github.com}
    \item Create a remote repository in \url{https://github.com/yourgit/proj.git}
    \item \TT{git remote add origin https://github.../proj.git}
\end{enumerate}


%%%%%%%%%%%%%%%%%%%%%%%%%%%%%%%%%%%%%%%
\section{Stash}
%%%%%%%%%%%%%%%%%%%%%%%%%%%%%%%%%%%%%%%%
\href{https://opensource.com/article/21/4/git-stash}{A practical guide to using the git stash command}


%%%%%%%%%%%%%%%%%%%%%%%%%%%%%%%%%%%%%%%
\subsection{Removing Remote URL}
\begin{flushleft}\begin{tabularx}{\textwidth}{l|X}
        \TT{git remote -v}        & Views the current remote                  \\
        \TT{git remote rm}        & Removes a remote URL from your repository \\
        \TT{git remote rm master} &
    \end{tabularx}\end{flushleft}


%%%%%%%%%%%%%%%%%%%%%%%%%%%%%%%%%%%%%%%
\subsection{Pushing}
\begin{flushleft}\begin{tabularx}{\textwidth}{l|X}
        \TT{git push -\:-u origin master} & Sends local changes to remote repository (\emph{origin})
    \end{tabularx}\end{flushleft}


%%%%%%%%%%%%%%%%%%%%%%%%%%%%%%%%%%%%%%%
\subsection{Pulling}
\begin{flushleft}\begin{tabularx}{\textwidth}{l|X}
        \TT{git pull origin master} & Pull-down any new changes (by collaborators etc.) from the remote repo.
    \end{tabularx}\end{flushleft}

%%%%%%%%%%%%%%%%%%%%%%%%%%%%%%%%%%%%%%%
\subsection{Example: working with git and GitHub}
\begin{enumerate}\packed
    \item \TT{mkdir D:/proj}
    \item \TT{echo "\# main.tex" >> D:/proj/README.txt}
    \item \TT{cd D:/proj}
    \item \TT{GIT init}
    \item \TT{git add README.txt}
    \item \TT{git commit -m "first commit"}
    \item \TT{remote add origin https://github.com/BehN/Git-Help-LaTeX.git}
    \item \TT{git push -u origin master}
\end{enumerate}

%%%%%%%%%%%%%%%%%%%%%%%%%%%%%%%%%%%%%%%
