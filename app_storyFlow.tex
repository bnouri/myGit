%%%%%%%%%%%%%%%%%%%%%%%%%%%%%%%%%%%%%%%%%%%%%%%%%%%%%%%%%%%%%%%%%%%%%%%%%%%%%%%%%%%
\section{Appendix: Reviewing some concepts with a descriptive style} \label{app:details}
\nl \fcolorbox{lightgray}{lightgray}{\textbf{Differences:}}
\nl Uh oh, looks like there have been some additions and changes to the cat family. Let's take a look at what is different from our last commit by using the \TT{git diff} command. In this case we want the diff of our most recent commit, which we can refer to using the HEAD pointer.
\nl\TT{git diff HEAD}

\bigskip
\nl \fcolorbox{lightgray}{lightgray}{\textbf{Staged Differences:}}
\nl Another great use for diff is looking at changes within files that have already been staged. Remember, staged files are files we have told git that are ready to be committed. Let's use git add to stage octofamily\/octodog.txt, which I just added to the family for you.
\nl\TT{git add octofamily/octodog.txt}

\nl Good, now go ahead and run git diff with the --staged option to see the changes you just staged. You should see that octodog.txt was created.
\nl \TT{git diff --staged}

\bigskip
\nl \fcolorbox{lightgray}{lightgray}{\textbf{Resetting the Stage:}}
\nl So now that octodog is part of the family, cat is all depressed. Since we love cat more than octodog, we'll turn his frown around by removing octodog.txt.
\nl You can unstage files by using the git reset command. Go ahead and remove octofamily/octodog.txt.
\nl\TT{git reset octofamily/octodog.txt}

\bigskip
\nl \fcolorbox{lightgray}{lightgray}{\textbf{Undo:}}
\nl git reset did a great job of unstaging octodog.txt, but you'll notice that he's still there. He's just not staged anymore. It would be great if we could go back to how things were before octodog came around and ruined the party.
\nl Files can be changed back to how they were at the last commit by using the command: git checkout -- <target>. Go ahead and get rid of all the changes since the last commit for cat.txt
\nl \TT{git checkout -- cat.txt}

\bigskip
\nl \fcolorbox{lightgray}{lightgray}{\textbf{Removing:}}
\nl Ok, so you're in the clean\_up branch. You can finally remove all those pesky cats by using the git rm command which will not only remove the actual files from disk, but will also stage the removal of the files for us.
\nl You're going to want to use a wildcard again to get all the cats in one sweep, go ahead and run:
\nl\TT{git rm '*.txt'}
\nl Removing one file is great and all, but what if you want to remove an entire folder? You can use the recursive option on git rm: 
\nl \TT{git rm -r folder\_of\_cats} 
\nl This will recursively remove all folders and files from the given directory.


\bigskip
\nl \fcolorbox{lightgray}{lightgray}{\textbf{Committing Branch Changes:}}
\nl Now that you've removed all the cats you'll need to commit your changes. Feel free to run git status to check the changes you're about to commit.
\nl\TT{git commit -m "Remove all the cats"}

\bigskip
\nl \fcolorbox{lightgray}{lightgray}{\textbf{Switching Back to master:}}
Great, you're almost finished with the cat... er the bug fix, you just need to switch back to the master branch so you can copy (or merge) your changes from the clean\_up branch back into the master branch.
\nl Go ahead and checkout the master branch:
\nl \TT{git checkout master}

\bigskip
\nl \fcolorbox{lightgray}{lightgray}{\textbf{Preparing to Merge:}}
\nl Alright, the moment has come when you have to merge your changes from the clean-up branch into the master branch. Take a deep breath, it's not that scary.
\nl We're already on the master branch, so we just need to tell Git to merge the clean\_up branch into it:
\nl\TT{git merge clean\_up}

\bigskip
\nl \fcolorbox{lightgray}{lightgray}{\textbf{Keeping Things Clean:}}
\nl You just accomplished your first successful bugfix and merge. All that's left to do is clean up after yourself. Since you're done with the clean\_up branch you don't need it anymore.
\nl You can use git branch -d <branch name> to delete a branch. Go ahead and delete the clean\_up branch now:\par
\nl \TT{git branch -d clean\_up}

\bigskip
\nl \fcolorbox{lightgray}{lightgray}{\textbf{git merge:}}
\nl When you’re done working on a branch, you can merge your changes back to the master branch, which is visible to all collaborators. git merge cats would take all the changes you made to the “cats” branch and add them to the master.
\nl\TT{git merge}